\documentclass{article}
\usepackage [utf8]           {inputenc}
\usepackage [greek, english] {babel}
\usepackage                  {alphabeta}
\usepackage{graphicx}
\graphicspath{ {images/} }
\usepackage[LGR, T1]{fontenc}
\usepackage{blindtext}
\usepackage[labelformat=empty]{caption}
\usepackage[raggedright]{titlesec}


\title{
    {\huge Διπλωματική Εργασία:\\~\\
    \Large	Προσαρμογή διαδικτυακού περιεχόμενου και δυναμικές ειδοποιήσεις με βάση τις συνήθειες πλοήγησης των χρηστών}
    \\~\\
    \includegraphics{images/University_of_Patras.jpg}\\
    {Πανεπιστήμιο Πατρών}\\
    {Τμήμα Μηχανικών Ηλεκτρονικών Υπολογιστών και Πληφορορικής}
}


\author{Φοιτητής: Γεώργιος Κορμπάκης\\
        Αριθμός Μητρώου: 5006}
%\date{\today}

%\titleformat{\paragraph}
%{\normalfont\normalsize\bfseries}{\theparagraph}{1em}{}
%\titlespacing*{\paragraph}
%{0pt}{3.25ex plus 1ex minus .2ex}{1.5ex plus .2ex}
\setcounter{secnumdepth}{5}
\titleformat{\paragraph}[hang]{\normalfont\normalsize\bfseries}{\theparagraph}{1em}{}
\titlespacing*{\paragraph}{0pt}{3.25ex plus 1ex minus .2ex}{1.5em}

\begin{document}
\clearpage\maketitle
\thispagestyle{empty}

\newpage
\setcounter{page}{1}


\section{Εισαγωγή}
\subsection{Σύντομη περιγραφή της διπλωματικής εργασίας}
Ο σκοπός αυτής της εργασίας είναι να μελετήσει το πώς και κατά πόσο ανταποκρίνονται οι αναγνώστες ενός ιστοτόπου, στις -προσαρμοσμένες βάσει της πρότερης πλοήγησής τους στον ιστότοπο- ειδοποιήσεις που λαμβάνουν στο κινητό τους τηλέφωνο όταν δημοσιεύεται νέο περιεχόμενο. \\

Η μελέτη βασίστηκε στην αναταπόκριση του αναγνωστικού κοινού ενός ελληνικού ιστοτόπου, ο οποίος ασχολείται με τη μουσική ειδησεογραφία. \\

Για να πραγματοποιηθεί η μελέτη, δόθηκε η ευκαιρία στο αναγνωστικό κοινό να δημιουργήσει λογαριασμό «Subscriber» στον ιστότοπο και να εγκαταστήσει την εφαρμογή στο smartphone του. Κατόπιν, δημιουργήθηκαν δύο διαφορετικοί αλγόριθμοι οι οποίοι, βασιζόμενοι στο περιεχόμενο των άρθρων που διαβάζει περισσότερο ο κάθε χρήστης, κατηγοριοποιούν το ενδιαφέρον του ανάλογα με καλλιτέχνες ή μουσικά είδη σε τρεις (3) κατηγορίες:
\begin{enumerate}
  \item Υψηλό ενδιαφέρον.
  \item Μέτριο ενδιαφέρον.
  \item Ελάχιστο/καθόλου ενδιαφέρον.
\end{enumerate}
Με αυτόν τον τρόπο, κατά τη δημοσίευση ενός νέου άρθρου, οι αλγόριθμοι λαμβάνουν απόφαση για αποστολή ειδοποίησης (περιπτώσεις 1 και 2 - διαφορετικού τύπου ειδοποίηση για κάθε περίπτωση) ή όχι (περίπτωση 3), βασιζόμενοι στο περιεχόμενο του άρθρου που μόλις δημοσιεύθηκε. \\
Τα αποτελέσματα και η λογική των δύο αλγορίθμων παρουσιάζονται παρακάτω.

\subsection{Εισαγωγικές έννοιες}
\subsubsection{Παγκόσμιος Ιστός}
Ο Παγκόσμιος Ιστός είναι ένα ανοιχτό σύστημα διασυνδεδεμένων πληροφοριών και πολυμεσικού περιεχομένου, που επιτρέπει στους χρήστες του Διαδικτύου να αναζητήσουν πληροφορίες μεταβαίνοντας από το ένα έγγραφο στο άλλο.\\~\\
Κάθε δίκτυο-δομική μονάδα του Διαδικτύου αποτελείται από συνδεδεμένους υπολογιστές σε τοπικό επίπεδο, για παράδειγμα το δίκτυο υπολογιστών των κεντρικών γραφείων μιας εταιρείας. Αυτά τα δίκτυα με τη σειρά τους συνδέονται σε ευρύτερα δίκτυα, όπως εθνικά και υπερεθνικά. Το ευρύτερο δίκτυο στον κόσμο λέγεται «Παγκόσμιος Ιστός». Είναι μοναδικό (δηλαδή δεν υπάρχουν παραπάνω από ένα δίκτυα υπολογιστών παγκόσμιας κλίμακας), και συμπεριλαμβάνεται τόσο τα γήινα δίκτυα, όσο και τα δίκτυα των δορυφόρων της και άλλων διαστημικών συσκευών που είναι συνδεδεμένα σε αυτό.
\newpage
Η τεχνολογία του ιστού δημιουργήθηκε το 1989 από τον Βρετανό Timothy John Berners-Lee, που εκείνη την εποχή εργαζόταν στον Ευρωπαϊκό Οργανισμό Πυρηνικών Ερευνών (CERN) στην Γενεύη της Ελβετίας. Το όνομα που έδωσε στην εφεύρεσή του ο ίδιος ο Lee είναι «World Wide Web», όρος γνωστός στους περισσότερους από το «www». Αυτό που οδήγησε τον Lee στην εφεύρεση του Παγκόσμιου ιστού ήταν το όραμά του για ένα κόσμο όπου ο καθένας θα μπορούσε να ανταλλάσσει πληροφορίες και ιδέες άμεσα προσβάσιμες από τους υπολοίπους. Το σημείο στο οποίο έδωσε ιδιαίτερο βάρος ήταν η μη ιεράρχηση των διασυνδεδεμένων στοιχείων. Οραματίστηκε κάθε στοιχείο, κάθε κόμβο του ιστού ίσο ως προς την προσβασιμότητα με τα υπόλοιπα. Αν σκεφτεί, όμως, κανείς τον βαθμό ιεράρχησης με τον οποίο λειτουργούν οι μηχανές αναζήτησης του ιστού, όπως για παράδειγμα το Google, γίνεται εύκολα κατανοητό ότι στην πράξη κάτι τέτοιο δεν συμβαίνει, τουλάχιστον στον βαθμό που το είχε οραματιστεί ο Lee.

\subsubsection{Smartphones}
Η βιομηχανία των κινητών τηλεφώνων γνώρισε πολύ μεγάλη ανάπτυξη όταν εγκαθιδρύθηκε ο όρος των smartphones. Πρόκειται για συσκευές οι οποίες ενσωματώνουν τόσο τις δυνατότητες ενός απλού τηλεφώνου, δηλαδή την πραγματοποίηση κλήσεων και την αποστολή γραπτών μηνυμάτων, όσο και δυνατότητες οι οποίες, μέχρι πρόσφατα, παρέχονταν μόνο μέσω ηλεκτρονικών υπολογιστών. Πλέον, μέσω των smartphones, δίνεται η δυνατότητα για πλοήγηση στο Διαδίκτυο, επικοινωνία μέσω εικόνας, ήχου και γραπτού λόγου μέσω αυτού, αναπαραγωγή πολυμέσων (π.χ. αρχεία μουσικής και βίντεο), ακόμα και εργασία διαμέσου ειδικά διαμορφωμένων περιβαλλόντων. Ταυτόχρονα, μέσω μίας ευρύτατης γκάμας εφαρμογών που είναι διαθέσιμες, είτε δωρεάν είτε όχι, ο κάθε χρήστης μπορεί να πραγματοποιήσει μια σειρά ενεργειών παντός τύπου, από διασκέδαση μέχρι ενημέρωση, επεξεργασία δεδομένων και λοιπά.\\~\\
Κατά συνέπεια, ο κόσμος που χρησιμοποιεί smartphones όλο και αυξάνεται, όπως φανερώνεται και στο παρακάτω γράφημα:
\begin{figure}[h]
  \caption{Χρήστες κινητών τηλεφώνων από το 2009 έως το 2018}
  \centering
  \includegraphics[width=0.6\textwidth]{Global_mobile_users.png}
\end{figure}

\paragraph{Android}
Το Android είναι ένα λειτουργικό σύστημα για κινητά τηλέφωνα (πλέον και για άλλες συσκευές όπως tablets, smartwatches) που δημιουργήθηκε από την Google. Η δημιουργία του ξεκίνησε το 2003 και βασίστηκε σε πυρήνα Linux και σκοπός του ήταν να ανταγωνιστεί τα λειτουργικά συστήματα Symbian και Microsoft Windows Mobile.\\
Οι δυνατότητες και κυρίως οι προοπτικές του Android δεν άργησαν να φανούν, με αποτέλεσμα το 2005 η Google να εξαγοράσει την εταιρεία που ξεκίνησε την ανάπτυξή του, Android Inc., έναντι 50 εκατομμυρίων δολαρίων. \\
Η πρώτη έκδοση του Android κυκλοφόρησε τον Σεπτέμβριο του 2008 και έκτοτε έχουν πραγματοποιηθεί πάρα πολλές αναβαθμίσεις, καθώς η ευκολία χρήσης που παρέχει, σε συνδυασμό με τη φορητότητά του αλλά και τις διαρκώς αυξανόμενες και βελτιούμενες δυνατότητες που προσφέρει, το έχουν κάνει το πιο ευρέως χρησιμοποιούμενο λειτουργικό σύστημα κινητών συσκευών. Ταυτόχρονα, όλα αυτά κίνησαν τόσο πολύ το ερευνητικό ενδιαφέρον, που πλέον βλέπουμε το Android να χρησιμοποιείται ακόμα και σε οικιακές συσκευές, αυτοκίνητα και πολλά ακόμα.\\
\begin{figure}[h]
  \caption{Ποσοστό των σελίδων που άνοιξαν μέσω κινητών συσκευών από το 2009 έως το 2018:}
  \centering
  \includegraphics[width=1\textwidth]{Perceptage_of_all_global_web_pages_served_to_mobile_phones_2009-2018.png}
\end{figure}
\paragraph{Mobile Εφαρμογές}
fff
\end{document}